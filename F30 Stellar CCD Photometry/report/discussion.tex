\section{Discussion}
We were able to verify the linearity up to a very high accuracy and observe the dynamical range of the CCD. We learned about some basic data reduction techniques, such as bias subtraction and flat-field correction. We observed the dark current and looked at its temperature dependency. We concluded the cooling process is detrimental in reducing the dark current, which is otherwise a natural lower bound of  the dynamical range and is deteriorating the sensitivity of the system by its stochastic nature.
\vspace{2mm}\\
The band gap determination has many errors sources which we didn't take into account, like the measurement of the temperature being not that accurate as well as the temperature dependency of the band gap. The decision to estimate an error by varying fit ranges made us overestimate the error given by the fit process by a factor of 10 (3 meV -> 40 meV). But considering the obvious deviation in the lower energies one could pose the question why we didn't consider the energy dependency from the band gap to get a more accurate plot. But if we try to fit with $\alpha$ and $\beta$ as parameter, it would become a very bad fit cause the parameters would correlated too much.
\vspace{2mm}\\
As already mentioned several times, the initial intention was to make an observation with the KING telescope of a globular cluster. If the weather would have been fitting, we would have used the data reduction we now just applied exemplary on some flat surface area. Afterwards analysing the images with the techniques performed on the archive pictures would have resulted in a CMD from our measurement, which would have been a nice way to finish this lab course.
\vspace{2mm}\\
To compare our results we read the reference paper \cite{rochau2007star}. As stated in the paper, the data are collected within the HST Program GO-10248, observed using the Wide-Field Channel (WFC) of ACS, centered on the association NGC 346, in the broad-band filters F555W(V-filter) and F814W (I-filter). With those information we were able to extract the zeropoints from the ACS zeropoint calculator \cite{HST_zeropoints} provided by Space Telescope Science Institute and compare those zeropoints to the ones we calculated. The deviation can be explained by us not counting all photons in the pictures for a standard star or in general an inconsistent way of approximating the areas of the photons of a star.\\