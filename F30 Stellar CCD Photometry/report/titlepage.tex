\begin{center}
	\makeatletter
	\thispagestyle{empty}
	\null\vspace{15mm}
	\Huge\textbf{\@title} \\
	\vspace{14mm}
	\Large  \textbf{\@author} \\
	\large supervised by: Stephan Stock\textbf{} \\
	\vspace{4mm}
	\large on March $25^{\text{th}} - \ 27^{\text{th}}$ 2019 \\
	\vspace{15mm}
	\large handed in as long report on: \  \@date \\ 
	\makeatother
	\vspace{20mm}
	\section*{Abstract}
\end{center}
This experiment has been performed as part of the advanced lab course for physics students (FP) at the University of Heidelberg.
\vspace{5mm}\\
The theoretical and experimental basics needed for the understanding of the conducted measurements and analysis is introduced and important concepts of the CCD are presented and discussed.
\vspace{5mm}\\
The characteristics of the CCD used in the KING telescope on the Königstuhl, Heidleberg, are analysed using data we recorded.
\vspace{5mm}\\
Additionally we will apply some basic data reduction techniques to our data and see the direct consequences of these.
\vspace{5mm}\\
Furthermore we used archival images from the Hubble Space Telescope (HST) to produce a color magnitude diagram (CMD) to estimate the distance, age, and metallicity of a cluster.
\vspace{5mm}\\
Finally if the weather allows it, we will make our own observation of a cluster and analyse it with such color magnitude diagram.\\