\section{Discussion}
The fits from the .vi program gave us errors of fit parameters for $T_2$  of around 2\%, but the difference between relaxation time measured by spin-echo and by Carr-Purcell method is around 7\%. In that sense we can see that molecular diffusion and inhomogeneities of magnetic field are major sources of systematic errors. And since it's still hard to estimate their effects in Carr-Purcell method, also it should be considered that some of the measurements had very few data points, it leads to the question how convincing such a fit error would be.
\vspace{3mm}\\
For p-xylol and toluol, in the substances determination by chemical shift, we expected the same peak positions with different amplitude, since they have similar NMR-active components. But as seen in table \ref{table2}, toluol's peak experienced a relative large shift compared to the other substances. For example the peak for CH$_3$ for p-xylol and acetic acid appear both around 2.0, but at 2.6 for toluol. The same could also be observed for the 3rd peaks, the benzol.
\vspace{5mm}\\
Unlike other spectroscopic techniques, NMR is unaffected by sample color and surface properties. Hence it is perfectly suited for many fields, e.g. food industry, healthcare industry etc., because it is a non destructive and non-invasive measurement that requires no sample preparation \cite{bruker}.\\
Although in this experiment we use NMR to differentiate several substances, in reality it is in some cases not practical. Since, as shown in figure \ref{shi1}, particular peaks can be referred as several parts and as long as one doesn't already have a general idea about the substance, it can be near impossible to find out what substance one are dealing with alone off those resonances. 
\vspace{3mm}\\
Some modern applications of NMR are for example the spectroscopy of $^{31}$P, which is an essential part of living cells, and Magnetic Resonance Imaging (MRI), which is widely used for medical purposes in modern day and age.\\