\section{Introduction}\label{intro}
\subsection{Basics of Nuclear Magnetic Resonance}\label{basics}
For a given Nuclei, whose spin $\vec{J}$ is unequal to 0, one can calculate it's magnetic dipole moment. 
\begin{equation}
	\label{1}
	\vec{\mu} = \hbar \gamma \vec{J}
\end{equation}
Where $\gamma$ is the gyromagnetic factor. In this experiment we will look at exclusively protons $\gamma_{proton} = 2.6752 \cdot 10^8 sec^{-1} Tesla^{-1}$.
\vspace{5mm} \\
The magnetization of N nuclei can be obtained, by summing over all nuclei per unit volume
\begin{equation}
	\label{2}
	\vec{M} = \dfrac{1}{V} \sum^{N}_{i=1} \mu_i
\end{equation}
In an external magnetic field the magnetic dipoles will realign themselves parallel or antiparallel to the direction of the magnetic field. The occupation number of the 2 states $N_{\pm}$, parallel $N_{+}$ and antiparallel $N_{-}$ , follow Dirac statistic, but they can be approximated with a Boltzmann distribution in this case 
\begin{equation}
	\label{3}
	N_{\pm}=N_0 e^{-\frac{E_0 \pm \Delta E}{kT}}
\end{equation}
with $N_0$ being a normalization factor. The parallel state is energetically favourable, hence $N_{+} > N_{-}$. 
\vspace{3mm} \\
If we insert \eqref{3} into \eqref{2} and use that $N = N_+ + N_-$ we can derive
\begin{equation}
	\label{4}
	\vec M =\frac{|\mu| N}{V} \sinh(\frac{|\mu| B}{kT}) \vec{e_z}
\end{equation}\\
You can expand \eqref{4} with the assumption of a weak field $(\mu B << kT)$, and derive Curie's law where $M \sim \frac{\vec{B_{0}}}{T}$. 
\vspace{5mm} \\
In general, the magnetization can have an arbitrary direction relative to the external field. It can be generated by applying a high frequency pulse $\omega_{HF}$ to the ground state. In the following we decompose it into the components $\vec{M_{\parallel}}$ parallel and $\vec{M_{\perp}}$ perpendicular to the external field. 
\vspace{3mm} \\
The Magnetic dipole interacts with an external magnetic field $\vec{B_{0}}$ 
\begin{equation}
	\label{5}
	\Delta E = - \vec{\mu}\cdot \vec{B_{0}}
\end{equation}
and as a result the general state of magnetization will dissipate it's excitation energy and reach the ground state asymptotically on a characteristic time scale.
\vspace{5mm} \\
The interaction results in a torque
\begin{equation}
	\label{6}
	\vec{\tau} = \vec{M} \times \vec{B_{0}}
\end{equation}
and since $\vec{B_{0}}$ is parallel to $\vec{M_{\parallel}}$, the torque only acts on $\vec{M_{\perp}}$. Without a relaxation processes, the rate of change is given by
\begin{equation}
	\label{7}
	\dfrac{d\vec{M_{\perp}}}{dt} = - \gamma \vec{M_{\perp}} \times \vec{B_{0}}
\end{equation}
This differential equation can be solved by an ansatz $\vec{M} = M_{\parallel}(cos(\omega_{L} t),sin(\omega_{L} t),0)$ with $\omega_{L}$ being the Larmor frequency 
\begin{equation}
	\label{8}
	\omega_{L} = \gamma B_{0}
\end{equation}
Consider the ground state magnetization $\vec{M}$ parallel to $\vec{B_{0}}$, pointing in the z-direction. A sinusoidal voltage of frequency $\omega_{HF}$ is applied on the along x.direction oriented coil, resulting in an induced magnetic field $B_{1}$ which longitudinally polarized along the x-direction. Then $\vec{M}$ precesses around the x-axis. During a time interval $\Delta t$ the angle $\alpha$ of the precession is then 
\begin{equation}
	\label{9}
	\alpha = \gamma B_{1} \Delta t
\end{equation}
If the time interval is chosen such that $\alpha = 90 \deg$, then $\vec{M}$ is rotated into a perpendicular component $\vec{M_{\perp}}$ along the y-axis. Such a pulse is called a $90 \deg$ pulse. Similarly we define $180 \deg$ pulse which results in magnetization antiparallel to the static field $\vec{B_{0}}$.
\vspace{4mm}\\
\subsection{Relaxation time}\label{relax}
Now we want to consider the relaxation process, which can be described with the Bloch equations. Here we introduce the rotating fame of the transverse magnetization, where the transverse magnetization is constant, if no relaxation processes takes place. The Bloch equations assume that the time evolution is dominated by a restoring force which is proportional to the deflection from equilibrium
\begin{equation}
	\label{10}
	\dfrac{dM_{\perp}(t)}{dt} = - \dfrac{M_{\perp}(t)}{T_{2}}
\end{equation}
\begin{equation}
	\label{11}
	\dfrac{dM_{\parallel}(t)}{dt} = - \dfrac{M_{\parallel}(t)-M_{0}}{T_{1}}
\end{equation}
where $T_{2}$ is the spin-spin relaxation time, $T_{1}$ the spin-lattice relaxation time and $M_{0}$ the ground state magnetization.
\vspace{5mm}\\
In the laboratory system we can now write the equations \eqref{7} as
\begin{equation}
	\label{12}
	\dfrac{dM_{\perp}(t)}{dt} = - \dfrac{M_{\perp}(t)}{T_{2}} + \gamma ( \vec{B} \times \vec{M})_{\perp}
\end{equation}
\begin{equation}
	\label{13}
	\dfrac{dM_{\parallel}(t)}{dt} = - \dfrac{M_{\parallel}(t)-M_{0}}{T_{1}} + ( \vec{B} \times 	\vec{M})_{\parallel}
\end{equation}
\subsubsection{Measuring Spin-spin relaxation $T_{2}$ with spin-echo method}
\begin{equation}
	\label{15}
	M_{\perp}(t) = M_{\perp}^{0}e^{-\frac{t}{T_{T_{2}}}}
\end{equation}
\begin{equation}
	\label{16}
	M_{\parallel}(t) = M_{0}(1-2e^{-\frac{t}{T_{T_{1}}}})
\end{equation}