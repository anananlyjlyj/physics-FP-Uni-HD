\section{Layout of the experiment}\label{layout-of-the-experiment}
The experiment is performed with Bruker minispec p20 in parts I and II,
Bruker minispec mq7.5 in part III.
\vspace{3mm}\\
The electronic unit of Bruker minispec p20 can generate 2 pulses,
i.e. Puls I, which should be set to 90$\deg$ pulse, and Puls II for 180$\deg$
pulse. Output signal can be set to be single or periodic with variable
periodicity. 4 Buttons allow us to produce pulse sequence Puls I - Puls
I, Puls II - Puls I, Puls I - Puls II, and Carr-Purcell sequence(Puls I
- Puls II - Puls II - Puls II - \ldots{} ).
\vspace{3mm}\\
The magnetic unit is connected to the p20 electronics by a cable, which
transmits the high frequency pulses $\omega_{HF}\) generated by the
electronics to the coil inside the magnet and, at the same time,
transmits the signal induced in the coil back to the electronics. The
coil signal $\omega_{L}$ is then mixed with the signal
$\omega_{HF}$. The mixing generate two signals, one with a frequency
of the sum and one with the frequency of the difference, i.e.~the
working frequency (on the order of a few hundred Hertz). One use the
working frequency for analysis. Note that $\omega_{HF}$ is fixed,
whereas $\omega_{L}$ can be changed by turning the screw of the
magnet.
\vspace{3mm}\\
The following picture shows an osizilloscope (up left), the electronic
unit of p20 (down left) and the magnetic unit of p20 (right).
%Figure 2.1: Bruker minispec p20, picture taken in labrotary
The Bruker minispec mq7.5 is equipped with a permanent magnetic field of
0.17 T and a measurement frequency of 7.5 MHz. Signals are generated by
the sensitive nuclei(hydrogen and fluorine) because of NMR and are
detected by the minispec.\\
%(ref:https://www.bruker.com/fileadmin/user\_upload/8-PDF-Docs/MagneticResonance/TD-NMR/minispec\_mqseries\_T137089.pdf)
The following picture shows a power supply(?) (left) and mq7.5 (right).
%Figure 2.2: Bruker minispec mq7.5, picture taken in labrotary
\vspace{3mm}\\
The experiment measurements, data acquisition and analysis of all 3
parts of this experiment was performed on two local computers in
laboratory. The software we used was LabVIEW, a systems engineering
software for applications that require test, measurement, and control
hardware and access to data.
%Figure 2.3: Computer used in part I and II, and probes for part II
%Figure 2.4: Example LabVIEW program in part III