\section{Layout of the experiment}\label{layout-of-the-experiment}
\subsection{Relaxation Time and Chemical Shift}\label{lay1}
For the first two parts of the this experiment we use a minispec p20 which produces both of the magnetic fields we need for the relaxation time and measuring the chemical shift. \\
\begin{figure}[h!]
	\centering
	\includegraphics[scale=0.18]{images/minispec_p20.jpg}
	\caption{The minispec p20, electronic unit on the left, magnetic unit on the right}
	\label{mini}
\end{figure} \\
The constant $\vec{B_0}$, which is proportional to $\omega_{L}$, can be adjusted by turning the screw sticking out of the Styrofoam as seen in Figure \ref{mini} . The $\omega_{HF}$ will be set by the electronic unit of the p20 seen on the left side of Figure \ref{mini}. An oscilloscope is also provided to measure the characteristic time for a $90\degree$ pulse, which is needed to calibrate the electronic unit of the p20 for the relaxation time measurements. The tubes with the substances get inserted into the magnetic unit parallel to the screw used to change $\omega_{L}$. A high-pressure air nozzle can be used to rotate the tube to ensure a evenly distributed density of the substance in in the tube, we will use it for the chemical shift part since there only 1 measurement for each tube will be required. \\
\begin{figure}[h!]
	\begin{subfigure}{0.32\textwidth}
	\includegraphics[width=0.9\linewidth ,height=4cm]{images/LabView1.png}
	\caption{Ga500 $T_2$ Spin-Echo}
	\label{Lab1}
	\end{subfigure}
	\begin{subfigure}{0.32\textwidth}
	\includegraphics[width=0.9\linewidth ,height=4cm]{images/LabView3.png}
	\caption{Ga500 $T_1$ Spin-Echo}
	\label{Lab3}
	\end{subfigure}
	\begin{subfigure}{0.32\textwidth}
	\includegraphics[width=0.9\linewidth, height=4cm]{images/LabView2.png}
	\caption{Chemical Shift}
	\label{Lab2}
	\end{subfigure}
	\caption{Example of the used Labviews}
	\label{exa}
\end{figure}\\
The data gets readout and analysed on a program written in Labview, the interface is shown in figure \ref{exa} via some example measurements.

\subsection{Imaging with NMR}\label{lay2}
We use a Bruker NMR analyzer 7.5 to generate the magnetic fields needed for 1D and 2D imaging. The readout gets fed into a computer where it gets analysed and can be later on displayed with a LabView .vi, examples of both can be seen in figure \ref{ImgLay}.
\begin{figure}[h!]
	\begin{subfigure}{0.5\textwidth}
	\includegraphics[width=0.9\linewidth ,height=4cm]{images/minispec_mq7_5.jpg}
	\caption{the Bruker NMR analyzer mq7.5}
	\label{Img1}
	\end{subfigure}
	\begin{subfigure}{0.5\textwidth}
	\includegraphics[width=0.9\linewidth ,height=4cm]{images/displaying_2d_image_labview_block.png}
	\caption{LabView .vi to display 2d image}
	\label{Img2}
	\end{subfigure}
	\caption{The NMR analyzer and LabView for 2D imaging}
	\label{ImgLay}
\end{figure}\\