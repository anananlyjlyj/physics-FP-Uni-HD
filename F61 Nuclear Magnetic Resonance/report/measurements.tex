\section{Measurements and evaluation}\label{measurements-and-evaluation}

\subsection{Measurement of relaxation time}\label{relaxation-time}

We measure $T_1$ and $T_2$ once with the spin echo method and $T_2$ again with the Carr-Purcell sequence for each Ga 500 and Ga 600.
\begin{table}[h!]
\centering
\begin{tabular}{c||c|c|c}
Substance  &  $T_{1,SE}$ & $T_{2,CP}$ & $T_{2,SE}$ \\
\hline
\hline
Ga 500 & 68.1 & 66.1 & 62.4 \\
\hline
Ga 600 & 99.2 & 84.0 & 78.6 \\
\end{tabular}
\caption{Summary of relaxation time measurements in $ms$}
\label{table1}
\end{table}\\
We can see 3 relations from these measurements: $T_{1} > T_{2}$ and $T_{2,CP} > T_{2,SE}$ for both substances as well as $T_{600} > T_{500}$ for each different method.\\
The first can explained by %TODO .
The second relation can be explained by looking at the effects of the Carr-Purcell method. This method minimizes the effects of the molecular diffusion and field inhomogeneities, which improves the precision of greater echo times.\\
The third relation can be attributed to the fact that the Ga 500 has higher concentration of hydrogen which leads to lower relaxation time .
\vspace{2mm}\\
From $\omega_L = 19.8$ MHz $\cdot 2\pi$ we can also calculate our external field $\vec{B_0}$. And with the characteristic time $\Delta t = 1.29\cdot 10^{-6}$ for a $90\degree$ pulse the solenoidal field $\vec{B_1}$ as well.
\begin{equation}\label{b_0}
	B_0 = \dfrac{\omega_L}{\gamma} = 0.48 T
\end{equation}
\begin{equation}\label{b_1}
	B_1 = \dfrac{\alpha}{\Delta t \gamma} = 4.3 mT
\end{equation}
where $\gamma$ was the gyromagnetic factor mentioned at the beginning for protons.
\subsection{Chemical shift}\label{chemical-shift}

We measure the peaks and determine which peak is the TMS. After that we calculate the shift with the given ppm of the .vi and appoint them to a substance from figure \ref{shi2} with the help of the provided cheat-sheet seen in figure \ref{shi1}. \\
\begin{table}[h!]
\centering
\begin{tabular}{c||c|c|c|c|c}
 & A+ & B+ & C+ & D+ & E+ \\
\hline
\hline
$\Delta$(2nd - TMS) & 2.2 & 2.1 & 2.0 & 3.9 & 2.6 \\
\hline
$\Delta$(3rd - TMS) & 3.9 & 6.9 & 11.6 & 6.3 & 7.5 \\
\hline
$\Delta$(4th - TMS) & 6.3 &  &  &  & \\
\hline
\hline
Substance & fluoroacetone & p-xylol & acetic acid & fluoroacetonitril & toluol \\
\end{tabular}
\caption{Summery of chemical shift in ppm}
\label{table2}
\end{table}\\
Even though the resonance frequency of Flour is a lot higher than the frequency we are using here, we can see the peaks caused by the Flour. They appear because the of the spin-spin interaction between Flour and the proton(hydrogen) in FCH$_2$ which lead to 2 different states for the electrons. For both D+ and A+ we can see each of these peaks at 3.9 and 6.3 on both spectra.
\vspace{2mm}\\
From the width of these peaks we can additionally calculate the energy resolution of this measurement as well as the energy difference between the 2 different states FCH$_2$ can be in depending on the spin-spin interaction of the electrons.
\begin{equation}\label{E_r}
	\Delta E_{res} = f_{FWHM} \cdot h = 19.9 Hz \cdot 4.136eVs = 8 \cdot 10^{-14}eV
\end{equation}
\begin{equation}\label{E_d}
	\Delta E_{dipole} = f_{\Delta F} \cdot h = 48 Hz \cdot 4.136eVs = 2 \cdot 10^{-13}eV
\end{equation}\\
\subsection{Imaging with NMR}\label{imaging-with-nmr}
First we operate the analyser in the 1D mode. We measure 15 mL oil in a tube and could observe the capillary effect of the liquid on the glass of the tube. Then we filled 50 mL of oil in a tube and measured it. There we could observe the boundaries of the analyser. They are caused by the inhomogeneity of the magnetic field if you are too far from the center of the coils. We then proceeded to fill an empty tube with 15 mm sand and 4 mm oil on top of it, which we observed in the analyser while the oil was seeping through the sand.