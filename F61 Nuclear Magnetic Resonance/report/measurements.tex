\section{Measurements and evaluation}\label{measurements-and-evaluation}

\subsection{Calibration high frequencypulses}\label{calibration-high-frequency-pulses}

\subsection{Measurement of relaxation time}\label{relaxation-time}

\subsubsection{Measurement T\_2 by spin echo method}\label{t_2-by-spin}

We insert probe into the magnet, select pulse sequence I-II and
\tau = 20 ms and active the pulse sequence. We use the LabVIEW
application "T2_meas.vi" on the computer.

For "single measurement" we enter the measuring time 400​ ms and
spin echo time $2\tau = 40​$ ms. To concentrate only on the essential
part around the maximum we choose del T for 15​ ms. Then we adjust 2
cursors on figure of Fourier transform of spectrum in
$2\tau \pm \Delta t$ range, so that only the single peak, noise
excluded, of amplitude is taken into calculation. Then we run ``many
measurements''for $10$ times. In that way we get the first data point
$(40, 5.75)$.

Then we do the same for $\tau = 40,60,80,100,120, 140$ ms.

In the end we click the button "fit" to draw the decay curve passing
through those data points and read the fit parameters.

We get the measurement result as Appendix \ldots{}

\subsubsection{Measurement T\_2 by Carr-Purcell sequence}\label{t_2-carr}

Similar to spin echo method, this time we use $tau = 20$ ms for probe
Ga 500 and enter the measuring time 800 ms. The maximum appears at
$t= \tau + 2 \tau = 60$ ms and then at t = 100, 140, 180 ms. (?)

We get the measurement result as Appendix \ldots{}

For probe Ga 600 we use \tau = 30 ms. The maximum appears at
$t=\tau+2\tau=90$ ms and then at t = 150, 210, 270 ms.

We get the measurement result as Appendix\ldots{}

Then we adjust the strength of the magnetic field to ensure the working
frequency is around 1000 Hz and do the measurements again. We get the
measurement result as Appendix\ldots{}

\subsubsection{Measurement T\_1}\label{measurement-t_1}

TODO

\subsubsection{Systematics of relaxation times for 2 probes}\label{2-probes}

According to the description in last section we do the measurements for
probes Gr 500 and Gr 600.

Table 3.1: Relaxation time measurements

\begin{longtable}[]{@{}llll@{}}
\toprule
\begin{minipage}[b]{0.22\columnwidth}\raggedright\strut
Probes\relaxation time\strut
\end{minipage} & \begin{minipage}[b]{0.21\columnwidth}\raggedright\strut
T\_2 by spin echo {[}ms{]}\strut
\end{minipage} & \begin{minipage}[b]{0.24\columnwidth}\raggedright\strut
T\_2 by Carr-Purcell {[}ms{]}\strut
\end{minipage} & \begin{minipage}[b]{0.21\columnwidth}\raggedright\strut
T\_1 by spin echo {[}ms{]}\strut
\end{minipage}\tabularnewline
\midrule
\endhead
\begin{minipage}[t]{0.22\columnwidth}\raggedright\strut
Gr 500\strut
\end{minipage} & \begin{minipage}[t]{0.21\columnwidth}\raggedright\strut
62.4\strut
\end{minipage} & \begin{minipage}[t]{0.24\columnwidth}\raggedright\strut
66.1\strut
\end{minipage} & \begin{minipage}[t]{0.21\columnwidth}\raggedright\strut
68.1\strut
\end{minipage}\tabularnewline
\begin{minipage}[t]{0.22\columnwidth}\raggedright\strut
Gr 600\strut
\end{minipage} & \begin{minipage}[t]{0.21\columnwidth}\raggedright\strut
78.6\strut
\end{minipage} & \begin{minipage}[t]{0.24\columnwidth}\raggedright\strut
84.0\strut
\end{minipage} & \begin{minipage}[t]{0.21\columnwidth}\raggedright\strut
99.2\strut
\end{minipage}\tabularnewline
\bottomrule
\end{longtable}

\subsection{Chemical shift}\label{chemical-shift}

\subsubsection{3.1 Substances identification}\label{substances-identification}

First we use a probe to get familiar with the measurement. We can see
that the intensity is distributed in broad range because of
inhomogeneity of the magnetic field. Its width is determined by? Then we
use the pressure air to put the probe into rotation and repeat the
measurement. We observe clearly some peaks, since in that way the
inhomogeneity is cancelled out.

We do the measurements for probes A+, B, B+, C, C+, D, D+ and E, E+
(probe A was broken in previous experiment). See Appendix \ldots{}(Name)
The LabVIEW program shows us the measured amplitude of signal changing
with time and plot its Fourier-transform in frequency space next to it.
In the lower part of the program we give the approximate range of each
peak we see in Fourier plot. The program makes a fit using the data and
calculate ppm from the position of each peak.(How did the program
calculate it?) Comparing two spectra for every substance, we find out
that the mixed reference substance in \emph{+ probe corresponds to the
most left peak in every spectrum of }+ probe. Subsequently we calculate
the difference of ppm given by the LabVIEW program.

Table 3.2: Chemical shift between 0. peak and other peaks

\begin{longtable}[]{@{}llllll@{}}
\toprule
relative to TMS peak(0. peak) & A+ & B+ & C+ & D+ & E+\tabularnewline
\midrule
\endhead
1. peak chemical shift & 2.2 & 2.1 & 2.0 & 3.9 & 2.6\tabularnewline
2. peak chemical shift & 3.9 & 6.9 & 11.6 & 6.3 & 7.5\tabularnewline
3. peak chemical shift & 6.3 & & & &\tabularnewline
\bottomrule
\end{longtable}

We use data on Figure 1.1 to match the chemical shift with possible
compounds. We associate five probes to the substances: (Do we need to
illustrate in detail?)

Table 3.3: Chemical substances identification

\begin{longtable}[]{@{}lllll@{}}
\toprule
A & B & C & D & E\tabularnewline
\midrule
\endhead
fluoroacetone & p-xylol & acetic acid & fluoroacetonitril &
toluol\tabularnewline
\bottomrule
\end{longtable}

where we use the intensity of the measured frequency response to
separate Toluol and p-xylol, which have the same chemical shifts because
of same compounds. Since -CH\_3 replaces one of the H-atom on benzene in
p-xylol, we expect two peaks which have less amplitude difference,
compared with two peaks of toluol.

Only the peak we measure for FCH\_2-CN do not correspond to what we
expect. We expect only one peak for 2 H-atoms in -CH\_2-, because the
resonance frequency for fluor is much higher than the range of our
measurement. However, we observe 2 peaks with almost same amplitude
close to each other. It can be explained by the spin-spin interaction of
F-atom with protons, one parallel and one antiparallel.

\subsubsection{Characteristics quantity of the experiment}\label{characteristics}

From the measured data we can infer some information about the magnetic
field inside the coil and about energy.

The external magnetic field \(B_0\) can be calculated from the Larmor
frequency \(\nu_L\), which is assumed to roughly equal the high
frequency \(\nu_{HF}=19.8 MHz\)(woher?), since their difference the
working frequency \(\nu_w\approx 500 Hz\) is relative small: \[
B_0 =\frac{\omega_L}{\gamma}=\frac{f_L \cdot 2 \pi}{\gamma}=\frac{19.8\cdot 10^6s^{-1}\cdot 2 \pi}{2.6752\cdot10^8 s^{-1}T^{-1}}\approx 0.48 T
\] The induced magnetic field \(B_1\) can be calculated from the pulse
time difference we set to achieve 90° pulse \(\Delta t=1.29\mu s\) ,
which we read from oscilloscope: \[
B_1 = \frac{\alpha}{\gamma\cdot\Delta t}=\frac{\pi/2}{2.6752\cdot10^8 s^{-1}T^{-1}\cdot 1.29\cdot10^{-6}s}\approx 4.3 mT
\] The energy resolution of our measurements can be calculated from the
FWHM of a peak \(\Delta f_{FWHM}=19.9 Hz\) taken from the first peak of
A+ using 2 cursors in the program: \[
\Delta E_{res} = \Delta f_{FWHM} \cdot h \approx 8\cdot10^{-14} eV
\] The energy difference for the different fluor states can be
calculated from the frequency difference of second peak and third peak
of A+ \(\Delta f=48 Hz\): \[
\Delta E_{dipol}=\Delta f \cdot h\approx 2\cdot10^{-13} eV
\]

\subsection{Imaging with NMR}\label{imaging-with-nmr}

\subsubsection{1d imaging}\label{d-imaging}

First we measure the image of 15mm oil (Appendix \ldots{}).

\subsubsection{2d imaging}\label{d-imaging-1}