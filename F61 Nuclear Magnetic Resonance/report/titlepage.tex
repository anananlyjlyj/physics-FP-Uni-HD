\begin{center}
	\makeatletter
	\thispagestyle{empty}
	\null\vspace{15mm}
	\Huge\textbf{\@title} \\
	\vspace{14mm}
	\Large  \textbf{\@author} \\
	\large supervised by: \textbf{Alena Harlenderova} \\
	\vspace{4mm}
	\large on Feburary $20^{\text{th}} / \ 21^{\text{th}}$ 2019 \\
	\vspace{15mm}
	\large handed in as short report on: \  \@date \\ 
	\makeatother
	\vspace{20mm}
	\section*{Abstract}
\end{center}
This experiment has been performed as part of the advanced lab course for physics students (FP) at Heidelberg University. 
\vspace{5mm}\\
The theoretical and experimental basics needed for the understanding of the conducted measurements is introduced and important concepts of nuclear magnetic resonance(NMR) are presented and discussed.
\vspace{5mm}\\
The characteristics of a NMR machine are analysed using data we recorded with the Bruker minispec p20 and the computer program LabView in the laboratory.
\vspace{5mm}\\
Furthermore we used Bruker minispec mq7.5 for getting 1d- and 2d-images of different subjects.
\vspace{5mm}\\
Lastly we present and discuss some applications of NMR in modern society.